To generate the `true data' which we will attempt to reconstruct, for $x\in[-L_{x},L_{x}]$, we use the initial conditions 
\begin{align*}
	\eta_{0}(x) = & \frac{1}{E}\sum_{m=-K+1,\neq0}^{K-1} e^{-\left(k_{m}-k_{d}\right)^{2}/2\tilde{\sigma}^{2}}e^{i(k_{m}x+\theta^{(\eta)}_{m})}\\
	q_{0}(x) = & \frac{1}{E}\sum_{m=-K+1,\neq0}^{K-1} e^{-\left(k_{m}-k_{d}\right)^{2}/2\tilde{\sigma}^{2}}e^{i(k_{m}x+\theta^{(q)}_{m})}\\
\end{align*}
where $\theta^{(\eta,q)}_{-m} = -\theta^{(\eta,q)}_{m}$, $k_{m}=\pi m/L_{x}$, and $E$ is the $l_{2}$ normalization condition
\[
E = \left(\sum_{m=-K+1,\neq0}^{K-1} e^{-\left(k_{m}-k_{d}\right)^{2}/\tilde{\sigma}^{2}}\right)^{1/2}.
\]
The phases $\theta^{(\eta,q)}_{m}$ are randomly chosen from a uniform distribution over the interval $[0,2\pi]$, and the $\neq0$ means that both the initial surface and velocity potential have zero spatial mean. Throughout the remainder of this section we choose the primary direction of propagation, $k_{d}$, so that $k_{d} = \pi/L_{x}$.  Thus, in the limit as $\tilde{\sigma}\rightarrow 0$, we go to the monochromatic initial conditions
\[
\eta_{0}(x) \sim \cos\left(\frac{\pi}{L_{x}}x +\theta^{(\eta)}\right), ~ q_{0}(x) \sim \cos\left(\frac{\pi}{L_{x}}x +\theta^{(q)}\right), ~ \tilde{\sigma} \sim 0.
\]
Note, our scaling choices ensure that each initial condition is $\mathcal{O}(1)$ in magnitude, which is consistent with the introduction of $\epsilon$ above as a measure of the effective surface-wave amplitude.  To generate the high-order numerics, we use a pseudo-spectral scheme with $K_{T}=2K=256$ modes and a 4th-order Runge--Kutta scheme with an integrating factor and time step $dt=.01$. We take $\epsilon=.1$ and $\mu=\sqrt{\epsilon}$ since this is consistent with the scaling choices made to derive classic nonlinear shallow water models such as the Korteweg--de Vries equation.  The number of terms used in the DNO expansion is $M=14$, which generally ensures machine precision accuracy for relatively long time evolutions.  

As for how we assimilate data, we set the number of ensemble members to be $N_{e}=100$.  To initialize the filter, we use the following initial conditions for $j=1,\cdots,N_{e}$   
\begin{align*}
	\eta^{(j)}_{0}(x) = & \frac{1}{E}\sum_{m=-K+1,\neq0}^{K-1} e^{-\left(k_{m}-k_{d}\right)^{2}/2\tilde{\sigma}^{2}}e^{i(k_{m}x+\theta^{(\eta,j)}_{m})}\\
	q^{(j)}_{0}(x) = & \frac{1}{E}\sum_{m=-K+1,\neq0}^{K-1} e^{-\left(k_{m}-k_{d}\right)^{2}/2\tilde{\sigma}^{2}}e^{i(k_{m}x+\theta^{(q,j)}_{m})}\\
\end{align*}
Note, by building our ensembles in this way, we are tacitly assuming that we know the initial power spectrum of the waves, leaving the only unknown to be the phases of each member of the ensemble.  This assumption is consistent with the use of results from spectral-wave modeling to approximate the behavior of essentially random-sea states.  
Further, per our scaling choices, we have that the time scale $\tau_{s} = L/\sqrt{gH}$ is given by
\[
\tau_{s}=\frac{L}{\sqrt{gH}} = \sqrt{\frac{H}{\epsilon g}} = \sqrt{H/1m}s = 10 s, ~H=100 m.
\]
If we assume a relatively conservative sampling rate of 10 times per second, then we could choose a non-dimensional sampling rate of up to $dt_{s} = .01$, equal above to the time step of the numerical solver.  However, this would not allow us to study the predictive power of our filtering process.  Thus, we choose $dt_{s}=.25$, or a cycle of $2.5s$ between data assimilation events.  We likewise use a 4th-order Runge--Kutta scheme with an integrating factor in order to perform the analysis to forecast update.  

\subsection*{Eularian Data Assimilation Results}


%%%%%%%%%%%%%%%%%%%%%%%%%%%%%%%%%%%%%%%%%%%%%%%%%%%%%%%%%%%%%%%%%%%%%%%%%%%%%%%%%%%